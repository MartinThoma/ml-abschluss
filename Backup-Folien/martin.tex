%!TEX root = Backup.tex
\section{Worum es geht}

\subsection{Die Aufgabe}

\begin{frame}{Die Aufgabe}
    \begin{itemize}
        \item \textbf{Eingabe}: Bilder, die von einer Kamera aus Fahrersicht
              aufgenommen wurden
        \item \textbf{Ausgabe}: Ein Bild gleicher Größe, wo jedes Pixel
              entweder schwarz ist (wenn der Klassifikator denkt es ist Straße)
              oder weiß ist (wenn dem) nicht so ist.
    \end{itemize}
\end{frame}

\begin{frame}{Die Daten}
    \href{http://www.cvlibs.net/datasets/kitti/eval_road.php}{KITTI Road Estimation dataset}
    \begin{itemize}
        \item Daten-Bilder der Größe $[1226, \dots, 1242] \times [370, \dots, 376]$,
              8-bit RGB
        \item Label-Bilder der selben Größe; 8-bit RGB mit 2 Farben
        \item 289 Trainingsbilder
        \item 290 Testbilder
    \end{itemize}
\end{frame}

\framedgraphic{Daten}{../images/umm_000000.png}
\framedgraphic{Labels}{../images/umm_road_000000.png}
\framedgraphic{Overlay}{../images/um_000000-overlay.png}